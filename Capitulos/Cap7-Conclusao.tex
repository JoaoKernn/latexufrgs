\chapter{Conclusão}

Este trabalho realizou um estudo de análise de risco de falha por erosão para uma tubulação de 3 polegadas de diâmetro nominal utilizada em sistemas hidráulicos de perfuração de poços petrolíferos. 
Para obtenção e comparação dos cenários provenientes da combinação das variáveis aleatórias que influenciam no desgaste erosivo, foram aplicados os seguintes métodos de análise de risco: (1) método da árvore de derivação com simulações computacionais CFD, (2) o método da árvore de derivação com metamodelo de superfície de resposta e (3) método de Monte Carlo com metamodelo de superfície de resposta.

Com intuito de obter as previsões quantitativas do desgaste erosivo, foram realizadas simulações fluidodinâmicas computacionais que propiciaram a modelagem de escoamentos multifásicos complexos com fluido e cascalhos de perfuração, com objetivo de obter as regiões mais propensas ao impacto dos particulados em determinadas condições de fluxo. Através da modelagem do escoamento, foi possível identificar o local de maior desgaste erosivo na tubulação curva e quantificar esse desgaste. Essa análise possibilitou a estimativa do tempo de falha do material em condições operacionais específicas, abrangendo as diversas condições requeridas para o estudo.

Com objetivo de realizar a triagem dos atributos que mais influenciam no Desgaste erosivo para as faixas estudadas das variáveis aleatórias, o método de planejamento de Plackett-Burman forneceu respostas semelhantes em relação à análise de sensibilidade. A análise de sensibilidade demonstrou apenas o impacto que a variação de níveis dos atributos causaram no desgaste erosivo, ficando a critério do pesquisador a seleção da quantidade de atributos para aplicação da análise de risco. Por outro lado, o Planejamento de Plackett Burman, forneceu os efeitos dos atributos mais significativos no desgaste erosivo através de testes estatísticos, sendo em geral, mais confiável para seleção dos atributos críticos, além de apresentar um critério de corte de variáveis com base na significância estatística.

A etapa de análise de risco foi viabilizada através da utilização de um metamodelo de regressão obtido por meio do Planejamento Estatístico Box Behnken. O desempenho desse metamodelo foi avaliado comparando suas previsões com os resultados obtidos por Simulação fluidodinâmica. Os resultados revelaram que o metamodelo apresentou uma capacidade satisfatória de prever os cenários de Desgaste erosivo, quando comparado aos resultados da simulação. Essa conclusão é suportada pelas métricas de avaliação, como R2, RMSE e MAE, que indicam uma boa capacidade de generalização do modelo, permitindo fazer previsões para cenários que não foram utilizados durante a construção do metamodelo. Além disso, foi constatado que o metamodelo se ajustou adequadamente às variações das respostas dos 81 cenários simulados pela Árvore de Derivação pela comparação dos percentis P10, P50 e P90 com redução de 69\% no número de simulações fluidodinâmicas. 

No estudo de análise de risco utilizando a abordagem do Método de Monte Carlo (MC), foram gerados 301.100 cenários acumulados para verificação da estabilização da resposta. Através da análise de Cov foi constatado equilibrio de resposta em aproximadamente 40.000 iterações, o que se mostraria impraticável do ponto de vista computacional, caso realizadas as simulações numéricas por CFD. Portanto, a utilização do metamodelo para o método de Monte Carlo foi fundamental para a obtenção dos cenários de erosão e realização da análise de risco. Dessa forma, o metamodelo fundamentado na Superfície de Resposta desempenhou um papel satisfatório ao substituir o simulador fluidodinâmico na obtenção das curvas de risco relacionadas ao desgaste erosivo. Ao oferecer confiabilidade estatística e reduzir significativamente a carga computacional, essa abordagem se estabelece como uma ferramenta confiável para a previsão do fenômeno de desgaste erosivo para análise de risco de um projeto petrolífero.

As diferenças percentuais entre os valores obtidos para os percentis da Árvore de derivação e Monte Carlo ocorreu devido às diferentes abordagens utilizadas para combinar os atributos aleatórios na geração dos cenários de erosão. O método de Monte Carlo oferece maior confiabilidade amostral, por apresentar um grande número de amostras das variáveis aleatórias e exercer uma análise de estabilização da resposta. Por outro lado, a Árvore de derivação completa propicia a geração de cenários diretamente a partir de simulações numéricas por CFD, sem a necessidade de utilização de um modelo de regressão para previsão dos dados. Os resultados indicaram que Monte Carlo e Árvore de derivação apresentam respostas similares de curva de risco de falha por erosão, desde que sejam utilizados metamodelos validados estatisticamente e sorteios amostrais representativos das distribuições das variáveis aleatórias, o que contribui para a tomada de decisão de um projeto.
 
Dessa forma, por meio da aplicação e análise de uma metodologia de análise de risco, é viável determinar o tempo estimado até a falha dos materiais empregados na perfuração de poços, considerando o espectro de probabilidades de falha decorrente do desgaste erosivo ao qual a tubulação do sistema hidráulico está sujeita sob condições operacionais incertas. Essa análise fornece uma base sólida para a tomada de decisões relacionadas à manutenção e substituição das tubulações antes do rompimento. Com base no tempo de falha indicado pela curva de risco e considerando o projeto de perfuração específico, é possível considerar a utilização de materiais mais resistentes à erosão, como os Flanges Alvo ("Target Flanges"), que possuem um custo de aquisição mais elevado em comparação com as tubulações curvas convencionais. No entanto, esses materiais oferecem uma maior durabilidade, o que pode resultar em economia ao reduzir a necessidade de substituições frequentes. Ao optar pelos Flanges Alvo, é possível obter uma maior vida útil operacional, o que compensa os custos iniciais mais altos e contribui para uma gestão mais eficiente dos recursos e do orçamento relacionados à manutenção das tubulações.
 
\section{Sugestões de trabalhos futuros}

Como sugestão de trabalhos futuros, seguem as recomendações:

a) Estudar outros métodos de obtenção de metamodelos como, por exemplo, Aprendizado de máquinas e Planejamento de Composto Central;

b) Realizar análise econômica de um projeto, a partir da Análise de risco, para avaliar a vantagem da utilização de material com maior resistência ao desgaste erosivo (\textlit{Target Flanges}) em substituição à curva de Aço-carbono;

c) Avaliar desgaste erosivo através de outros modelos, tais como: Finnie, DNV, Tulsa;

d) Realizar estudo de Simulação Fluidodinâmica transiente para obtenção do desgaste erosivo para realização da Análise de risco.

e) Obtenção de dados reais de perfuração em campo para caracterização amostral das variáveis aleatórias que influenciam no desgaste erosivo e realizar testes de aderência amostrais. 

f) Avaliar a correlação e interdependência entre as variáveis aleatórias.

g) Validação experimental das Simulações Numéricas e do metamodelo de Superfície de resposta para previsão do desgaste erosivo.
